\section{Изучение универсальных методов сжатия на примере короткого текста}
Задан текст: 
\begin{equation}
    \texttt{<<In\_order\_to\_get\_where\_you\_want\_to\_go,\_you\_first\_have\_to\_leave\_where\_you\_are>>}
\end{equation}

Требуется применить к нему универсальные методы сжатия и сопоставить результаты с некоторыми стандартными архиваторами.

\subsection{Двупроходное кодирование с использованием кода Хаффмена}
Для решения задачи была написана программа, посчитаны вероятности и кодовые слова. Они предоставлены в следующей таблице:

\renewcommand\arraystretch{1.0}
\begin{center}
    \begin{longtable}{|c|c|c|c|}
        \hline
        Буква&Частота буквы в тексте&Длина кодового слова&Кодовое слово \\
        \hline  \_&16&2&00\\  \hline o&8&3&010\\  \hline a&4&4&0110\\  \hline n&2&5&01110\\  \hline g&2&5&01111\\  \hline i&1&6&100000\\  \hline s&1&6&100001\\  \hline ,&1&6&100010\\  \hline I&1&6&100011\\  \hline f&1&6&100100\\  \hline d&1&6&100101\\  \hline y&3&5&10011\\  \hline e&10&3&101\\  \hline t&6&4&1100\\  \hline r&6&4&1101\\  \hline w&3&5&11100\\  \hline u&3&5&11101\\  \hline h&3&5&11110\\  \hline l&1&6&111110\\  \hline v&2&6&111111\\  \hline
    \end{longtable}
\end{center}

В итоге каждый символ текста был закодирован, согласно следующей таблице:
\begin{center}
    \begin{longtable}{|c|c|c|c|c|}
        \hline
        Буква&Вероятность буквы&Длина кодового слова&Кодовое слово&Длина сообщения \\
        \hline  I&1/75&6&111110&6\\  \hline  n&2/75&5&01111&11\\  \hline  \_&16/75&2&00&13\\  \hline  o&8/75&3&010&16\\  \hline  r&2/25&4&1101&20\\  \hline  d&1/75&6&100011&26\\  \hline  e&2/15&3&101&29\\  \hline  r&2/25&4&1101&33\\  \hline  \_&16/75&2&00&35\\  \hline  t&2/25&4&1100&39\\  \hline  o&8/75&3&010&42\\  \hline  \_&16/75&2&00&44\\  \hline  g&2/75&5&01110&49\\  \hline  e&2/15&3&101&52\\  \hline  t&2/25&4&1100&56\\  \hline  \_&16/75&2&00&58\\  \hline  w&1/25&5&10011&63\\  \hline  h&1/25&5&11110&68\\  \hline  e&2/15&3&101&71\\  \hline  r&2/25&4&1101&75\\  \hline  e&2/15&3&101&78\\  \hline  \_&16/75&2&00&80\\  \hline  y&1/25&5&11101&85\\  \hline  o&8/75&3&010&88\\  \hline  u&1/25&5&11100&93\\  \hline  \_&16/75&2&00&95\\  \hline  w&1/25&5&10011&100\\  \hline  a&4/75&4&0110&104\\  \hline  n&2/75&5&01111&109\\  \hline  t&2/25&4&1100&113\\  \hline  \_&16/75&2&00&115\\  \hline  t&2/25&4&1100&119\\  \hline  o&8/75&3&010&122\\  \hline  \_&16/75&2&00&124\\  \hline  g&2/75&5&01110&129\\  \hline  o&8/75&3&010&132\\  \hline  ,&1/75&6&100101&138\\  \hline  \_&16/75&2&00&140\\  \hline  y&1/25&5&11101&145\\  \hline  o&8/75&3&010&148\\  \hline  u&1/25&5&11100&153\\  \hline  \_&16/75&2&00&155\\  \hline  f&1/75&6&100010&161\\  \hline  i&1/75&6&100100&167\\  \hline  r&2/25&4&1101&171\\  \hline  s&1/75&6&100000&177\\  \hline  t&2/25&4&1100&181\\  \hline  \_&16/75&2&00&183\\  \hline  h&1/25&5&11110&188\\  \hline  a&4/75&4&0110&192\\  \hline  v&2/75&6&111111&198\\  \hline  e&2/15&3&101&201\\  \hline  \_&16/75&2&00&203\\  \hline  t&2/25&4&1100&207\\  \hline  o&8/75&3&010&210\\  \hline  \_&16/75&2&00&212\\  \hline  l&1/75&6&100001&218\\  \hline  e&2/15&3&101&221\\  \hline  a&4/75&4&0110&225\\  \hline  v&2/75&6&111111&231\\  \hline  e&2/15&3&101&234\\  \hline  \_&16/75&2&00&236\\  \hline  w&1/25&5&10011&241\\  \hline  h&1/25&5&11110&246\\  \hline  e&2/15&3&101&249\\  \hline  r&2/25&4&1101&253\\  \hline  e&2/15&3&101&256\\  \hline  \_&16/75&2&00&258\\  \hline  y&1/25&5&11101&263\\  \hline  o&8/75&3&010&266\\  \hline  u&1/25&5&11100&271\\  \hline  \_&16/75&2&00&273\\  \hline  a&4/75&4&0110&277\\  \hline  r&2/25&4&1101&281\\  \hline  e&2/15&3&101&284\\  \hline 
    \end{longtable}
\end{center}

В итоге получилось 284 бита на закодированное сообщение, $20 \cdot 8 = 160$ бит на кодовые слова и $20 \cdot 2 - 1 = 39$ бит на дерево Хаффмена. 
В итоге $284 + 160 + 39 = 483$ бита.

\subsection{Арифметическое кодирование}
Для адаптивного арифметического кодирования был использован A-алгоритм:
\begin{equation}
p_n(a) = \begin{cases}
            \frac{\tau_n(a)}{n + 1}, & \text{если $\tau_n(a) > 0$} \\
            \frac{1}{(n + 1) \cdot (M - M_n)}, & \text{если $\tau_n(a) = 0$}
        \end{cases}
\end{equation}
 
 
В следующей таблице приведен результат работы.

\begin{center}
    \begin{longtable}{|c|c|c|}
        \hline
        Буква&Текущая вероятность&G \\
         \hline  I&1/256&0.00390625\\ \hline  n&1/510&7.659313725490196e-06\\ \hline  \_&1/762&1.0051592815603932e-08\\ \hline  o&1/1012&9.932403967988075e-12\\ \hline  r&1/1260&7.882860292054028e-15\\ \hline  d&1/1506&5.234302982771599e-18\\ \hline  e&1/1750&2.991030275869485e-21\\ \hline  r&1/8&3.738787844836856e-22\\ \hline  \_&1/9&4.1542087164853956e-23\\ \hline  t&1/2490&1.668356914251163e-26\\ \hline  o&1/11&1.5166881038646937e-27\\ \hline  \_&1/6&2.527813506441156e-28\\ \hline  g&1/3224&7.840612613030881e-32\\ \hline  e&1/14&5.6004375807363436e-33\\ \hline  t&1/15&3.733625053824229e-34\\ \hline  \_&3/16&7.00054697592043e-35\\ \hline  w&1/4199&1.667193849945327e-38\\ \hline  h&1/4428&3.765117095630821e-42\\ \hline  e&2/19&3.9632811532956006e-43\\ \hline  r&1/10&3.9632811532956e-44\\ \hline  e&1/7&5.661830218993716e-45\\ \hline  \_&2/11&1.0294236761806755e-45\\ \hline  y&1/5635&1.82683882197103e-49\\ \hline  o&1/12&1.5223656849758584e-50\\ \hline  u&1/6100&2.4956814507800957e-54\\ \hline  \_&5/26&4.799387405346338e-55\\ \hline  w&1/27&1.777550890869014e-56\\ \hline  a&1/6804&2.612508657949756e-60\\ \hline  n&1/29&9.008650544654332e-62\\ \hline  t&1/15&6.005767029769554e-63\\ \hline  \_&6/31&1.1624065218908817e-63\\ \hline  t&3/32&1.0897561142727014e-64\\ \hline  o&1/11&9.906873766115467e-66\\ \hline  \_&7/34&2.039650481259067e-66\\ \hline  g&1/35&5.827572803597334e-68\\ \hline  o&1/9&6.475080892885927e-69\\ \hline  ,&1/8954&7.231495301413811e-73\\ \hline  \_&4/19&1.522420063455539e-73\\ \hline  y&1/39&3.9036411883475363e-75\\ \hline  o&1/8&4.8795514854344203e-76\\ \hline  u&1/41&1.1901345086425415e-77\\ \hline  \_&3/14&2.5502882328054463e-78\\ \hline  f&1/10363&2.460955546468635e-82\\ \hline  i&1/10560&2.330450328095298e-86\\ \hline  r&1/15&1.553633552063532e-87\\ \hline  s&1/10994&1.4131649554880228e-91\\ \hline  t&4/47&1.2026935791387425e-92\\ \hline  \_&5/24&2.505611623205714e-93\\ \hline  h&1/49&5.113493108583089e-95\\ \hline  a&1/50&1.022698621716618e-96\\ \hline  v&1/12138&8.425594181221106e-101\\ \hline  e&1/13&6.481226293247005e-102\\ \hline  \_&11/53&1.3451601740701331e-102\\ \hline  t&5/54&1.2455186796945678e-103\\ \hline  o&6/55&1.358747650575892e-104\\ \hline  \_&3/14&2.9116021083769112e-105\\ \hline  l&1/13509&2.155305432213274e-109\\ \hline  e&5/58&1.8580219243217877e-110\\ \hline  a&2/59&6.298379404480636e-112\\ \hline  v&1/60&1.0497299007467727e-113\\ \hline  e&6/61&1.0325212138492846e-114\\ \hline  \_&13/62&2.1649638354904353e-115\\ \hline  w&2/63&6.872901065049002e-117\\ \hline  h&1/32&2.147781582827813e-118\\ \hline  e&7/65&2.312995550737645e-119\\ \hline  r&2/33&1.4018154852955422e-120\\ \hline  e&8/67&1.6738095346812444e-121\\ \hline  \_&7/34&3.4460784537555035e-122\\ \hline  y&2/69&9.988633199291313e-124\\ \hline  o&1/10&9.988633199291313e-125\\ \hline  u&2/71&2.8136994927581164e-126\\ \hline  \_&5/24&5.861873943246077e-127\\ \hline  a&3/73&2.408989291744963e-128\\ \hline  r&5/74&1.6276954673952452e-129\\ \hline  e&3/25&1.953234560874294e-130\\ \hline 

    \end{longtable}
\end{center}

В итоге $G$ примерно равно:
\begin{equation}
\frac1{5119712808851725692816552456016762731640951109900195370850048610818595016807 \cdot 10^{54}}
\end{equation}

Ну и длина кода равна:

\begin{equation}
    l = \lceil -\log G \rceil + 1 = 432
\end{equation}

\subsection{Нумерационное кодирование}

\begin{center}
    \begin{longtable}{|c|c|c|}
    \hline
    Буква&Индекс буквы&Количество вхождений буквы в текст \\    
 \hline ,&44&1\\  \hline I&73&1\\  \hline \_&95&16\\  \hline a&97&4\\  \hline d&100&1\\  \hline e&101&10\\  \hline f&102&1\\  \hline g&103&2\\  \hline h&104&3\\  \hline i&105&1\\  \hline l&108&1\\  \hline n&110&2\\  \hline o&111&8\\  \hline r&114&6\\  \hline s&115&1\\  \hline t&116&6\\  \hline u&117&3\\  \hline v&118&2\\  \hline w&119&3\\  \hline y&121&3\\  \hline 
    \end{longtable}
\end{center}

Номер данной композиции равен:
$$
    3397120139502415624975349973851857840183453004248002394366454651971303032576
$$
Длина этого числа в двоичной записи 253 бита. Индекс заданной нам строки в отсортированном в лексикографическом порядке списке всех строк с заданной композицией:
$$
    62825702127366090376149724605382273591532344574336711158549954560000000000
$$
Длина этого числа в двоичной записи 248 бит. И длина сообщения $n = 75$, 7 бит. Итого $253+248+7=508$ бит.
\subsection{Алгоритм LZ-77}
