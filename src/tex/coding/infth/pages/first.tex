\section{Энтропия марковского источника и код Хаффмена}
Для марковского источника с заданной матрицей $P$ переходных вероятностей найти: $H(X)$,
$H(X|X^{\infty})$,
$H_2(X)$,
$H_n(X)$.
\renewcommand{\arraystretch}{2.0}
\begin{equation}
P=
\left[\begin{array}{ccc}
\frac13 & 0 & \frac23 \\
\frac14 & \frac 12 & \frac14 \\
0 & \frac12 & \frac12
\end{array}\right]
\end{equation}

Построить коды Хаффмена для ансамблей $X$, $X^2$.

\subsection{Вычисление энтропии}
Заметим, что заданная цепь Маркова эргодична, так как $P^k$ для $k \ge 2$ содержит только положительные элементы, поэтому существует предел:
\begin{equation}
    \mathbf{p} = \lim\limits_{n \to \infty} \mathbf{p_0}P^n
\end{equation}
такой что,
\begin{equation}
    \mathbf{p} = \mathbf{p} P
\end{equation}

Найдем $\mathbf{p}$. Зная что $\mathbf{p} = (p_1, p_2, p_3)$ и $\sum\limits_{i=1}^3{p_i} = 1$, составим систему уравнений:
\begin{equation}
\left\{
\begin{aligned}
\frac13p_1+\frac14p_2=p_1 \\
\frac12p_2+\frac12p_3=p_2 \\
\frac23p_1+\frac14p_2+\frac12p_3=p_3 \\
p_1+p_2+p_3=1
\end{aligned}
\right.
\end{equation}

Решая систему, получаем:

\begin{equation}
    \mathbf{p} = \left(\frac3{19}, \frac8{19}, \frac8{19}\right)
\end{equation}

Вычисляем энтропию:
\begin{equation}
H(X)=\sum\limits_{i=1}^3{-p_i\log{p_i}}=-\frac{3}{19}\log\frac{3}{19}-2\frac{8}{19}\log\frac{8}{19}\approx 1.0199
\end{equation}

В заданной цепи Маркова $H(X|X^\infty) = H(X|X)$, вычислим:
\begin{equation}
    \begin{split}
        &H(X|X^\infty)=H(X|X)=-\sum\limits_{i=1}^3\sum\limits_{j=1}^3{p(x_i, x_j)\log{p(x_j|x_i)}} \\
        &H(X|X^\infty) \approx 0.8301
    \end{split}
\end{equation}

Продолжим вычислением $H_2(X)$:
\begin{equation}
    \begin{split}
        &H_2(X)=\frac{H(X^2)}2 = -\frac12 \sum\limits_{i=1}^3\sum\limits_{j=1}^3{p(x_i, x_j) \log{p(x_i, x_j)}} \\
        &H_2(X) \approx 0.9250
    \end{split}
\end{equation}

И вычислим $H_n(X)$:
\begin{equation}
    \begin{split}
        &H_n(X)=H(X|X^n)+\frac1n(H(X)-H(X|X^s))=H(X|X)+\frac1n(H(X)-H(X|X)) \\
        &H_n(X)=0.8301+\frac1n(1.0199-0.8301)=0.8301+\frac{0.1898}n
    \end{split}
\end{equation}

\subsection{Построение кода Хаффмена}
